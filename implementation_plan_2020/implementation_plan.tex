\documentclass{article}

\usepackage[
    backend=bibtex, 
    natbib=true,
    style=alphabetic,
    citestyle=alphabetic,
    maxcitenames=1,
    maxbibnames=99
]{biblatex}

\usepackage[utf8]{inputenc} 
\usepackage[T1]{fontenc}
\usepackage{enumitem}
\usepackage{lmodern}
\usepackage{graphicx}
\usepackage{tikz}
\usetikzlibrary{calc}
\usetikzlibrary{positioning}
\usepackage{url}
\usepackage{rotating}
\usepackage{graphicx}            % For pdf/bitmapped graphics files
\usepackage[margin=3cm]{geometry}
\graphicspath{{images/}}
\DeclareGraphicsExtensions{.pdf,.png,.jpg,.jpeg}

\usepackage{hyperref}
\usepackage{todonotes}

\begin{document}


\begin{center}
  {\Huge \bfseries
    RoboCup Soccer: Humanoid League
    \vspace{\stretch{1}}\\
    \rule{\textwidth}{0.1cm}\\
    \vspace{0.5cm}
    Implementation plan proposal 2020\\
    \vspace{0.5cm}
    \rule{\textwidth}{0.1cm}\\
  }
  \vspace{\stretch{4}}
  {\bfseries \today}
\end{center}

\newpage

\section{Introduction}

For RoboCup 2020, we have multiple structural changes for the Humanoid League.
The Teen Size class will merge within both Kid and Adult and several new leagues
will be introduced.

The leagues we intend to run are:

\begin{description}[leftmargin=5em,style=nextline]
\item[HSCK:] Humanoid Soccer Competition Kid, (merge of Humanoid Kid Size and
  Humanoid Teen Size, will likely result in two divisions)
\item[HSCA:] Humanoid Soccer Competition Adult, (merge of Humanoid Adult Size
  and Humanoid Teen Size)
\item[HOCK:] {Humanoid Open Competition Kid}
\item[HOCA:] {Humanoid Open Competition Adult}
\item[HRD:] {Humanoid Research Demonstration}
\end{description}

In this document we detail the implementation of our proposal in terms of
trophies, certificates and impact on requirements for the venue.

\section{Number of teams per league}

The number of teams who registered for RoboCup 2019 and our expectations for
RoboCup 2020 are presented in Table~\ref{tab:nb_teams}.
It is particularly difficult to anticipate the number of application we will
have for the Humanoid Open Competitions and the Humanoid Research Demonstration.
However, we can mention the following elements:
\begin{itemize}
\item HOCK will not be opened if there are less than four teams registering.
\item For HOCA, we would like to be able to open it even if there are only 2
  teams registering.
  The number of teams participating to the RoboCup with adult size robots is
  currently too low to afford to lose candidates.
\item Since HRD is not a competitive event, but a research exhibition,
  it does not require a minimal amount of teams to be meaningful.
  \todo{If we add judging, this would not be valid}
\end{itemize}

\begin{table}[h]
  \centering
  \caption{\label{tab:nb_teams}Number of teams in 2019 and expected
    participation for 2020}
  \begin{tabular}{l | r | r}
    League & 2019 & 2020\\
    \hline
    HSCK: & 16 & 20-30\\ 
    HSCT: & 8 & \\ 
    HSCA: & 5 & 5-8\\ 
    HOCK: &  & 4-8\\
    HOCA: &  & 2-4\\ 
    HRD: &  & 5-10\\
    \hline
    \textbf{Total} & 29 & 36-60
  \end{tabular}
\end{table}


\section{Trophies and certificates}

The evolution of the number of trophies is presented in
Table~\ref{tab:trophies}\todo{Confirm 2019 numbers}.
The slight increase in the number of trophies is conditioned to having enough
teams registering for all the leagues, which would lead to a global increase of
the number of teams.
Therefore, the average number of trophies requested per team tend to decrease
with our proposal.
Since the \emph{Best Humanoid Award} is conditioned to obtaining a new sponsor,
we do not consider it here.

\begin{table}[h]
  \centering
  \caption{\label{tab:trophies}Trophies evolution}
  \begin{tabular}{l | r | r}
    League & 2019 & 2020\\
    \hline
    HSCK: Main tournament & 3 & 6\\ 
    HSCK: Technical Challenges & 1 & 1\\
    HSCK: Drop-In & 1 & 1\\
    HSCT: Main tournament & 2 & \\ 
    HSCT: Technical Challenges & 1 & \\
    HSCT: Drop-In & 1 & \\
    HSCA: Main tournament & 3 & 3\\ 
    HSCA: Technical Challenges & 1 & 1\\
    HSCA: Drop-In & 1 & 1\\
    HOCK: &  & 1\\
    HOCA: &  & 1\\ 
    HRD: &  & 0\\
    \hline
    \textbf{Total} & 14 & 15
  \end{tabular}
\end{table}


\todo{Should we include certificate list}

\section{Registration fees}

For the registration fee, we propose different approaches depending on the
league:

\begin{description}
\item[HSCK and HSCA:] We keep the existing scheme of having free team
  registration for new teams while all members have to register.
\item[HOCK and HOCA:] For RoboCup 2020, we propose that to have free team
  registration since both are new leagues.
  Then, we can keep the same scheme with free team registration for new teams.
  In both cases, members still have to register.
\item[HRD:] Since it is a competition free event with little requirements,
  we propose to use the following scheme for registration fees.
  Team registration is free and members who already registered for another
  RoboCup league do not need to pay additional fees.
  Individual who are not registered in any league still need to pay their own
  registration.
\end{description}

\section{Requirements for the venue}

Opening the new leagues has limited impact on the requirements for the venue.
We expect HOCK to share the fields with HSCK and HOCA to share fields with HSCA,
HRD will also happen on the regular soccer fields and will therefore not require
extra materials.
As is shown in table~\ref{tab:nb_teams}, we expect less teams in the new
leagues,
therefore there should be a reasonable increase in terms of number of
participants.

At RoboCup 2019, we managed to run succesfully the competition despite the late
announcement that we would have one field less than expected.
However, the organization of the first three days was heavily impacted by the
number of fields available:
\begin{itemize}
\item We had to use 45 minutes time slots instead of the usual one hour
  time slots.
\item On-site testing and practice for the teams was drastically reduced.
\item Kid Size semi-finals had to occur the last day of competition which is
  susceptible to decrease the quality of the finals.
\end{itemize}
In order to avoid similar issues, we definitely need more
Therefore, we definitely need more fields than what was available during RoboCup
2019 in order to run a successful event.
Table~\ref{tab:requirements} compares the organization of the RoboCup 2019 with
the expected elements for RoboCup 2020.
It takes into account that HRD occurs on the large field in order to make sure
there is sufficient space for the teams.

% Time slots for the first 3 days:
% - RR: Round-Robin
% - KO: Knock-out game
% - DI: Drop-In
% - TC: Technical Challenge
% 2019:
% - Kid time slots: 24RR + 12KO + 10DI + 4TC -> 50
% - Teen time slots: 12RR + 6KO + 8DI + 2TC -> 28
% - Adult time slots: 10RR + 2KO + 8DI + 2TC -> 22
% 2020:
% - HSCK time slots (8DivA,16DivB): 20DI + 36RR + 18KO + 6TC -> 80
% - HOCK time slots: 5-15DI
% - HSCA time slots: 10RR + 2KO + 8DI + 2TC -> 22
% - HOCA time slots: 4-8
% - HRD  time slots: 4-10
\begin{table}[h]
  \centering
  \caption{\label{tab:requirements}Venue requirements (time slots concern only
    the first three days)}
  \begin{tabular}{l|r|r|r|r|}
    & \multicolumn{2}{c}{2019} & \multicolumn{2}{c}{2020}\\
    & Time slots required & Nb fields & Time slots expected & Nb fields\\
    \hline
    Small field: $11m \cdot 8m$ & 78 & 4 & 85-95 & 5\\
    Large field: $16m \cdot 11m$ & 22 & 1 & 30-40 & 2\\
    % 4 persons working table: & 70 & 80-120
  \end{tabular}
\end{table}

\end{document}
