\documentclass{article}

\usepackage[
    backend=bibtex, 
    natbib=true,
    style=alphabetic,
    citestyle=alphabetic,
    maxcitenames=1,
    maxbibnames=99
]{biblatex}

\usepackage[utf8]{inputenc} 
\usepackage[T1]{fontenc}
\usepackage{enumitem}
\usepackage{lmodern}
\usepackage{graphicx}
\usepackage{tikz}
\usetikzlibrary{calc}
\usetikzlibrary{positioning}
\usepackage{url}
\usepackage{rotating}
\usepackage{graphicx}            % For pdf/bitmapped graphics files
\usepackage[margin=3cm]{geometry}
\graphicspath{{images/}}
\DeclareGraphicsExtensions{.pdf,.png,.jpg,.jpeg}

\usepackage{hyperref}
\usepackage{todonotes}

\begin{document}


\begin{center}
  {\Huge \bfseries
    RoboCup Humanoid League
    \vspace{\stretch{1}}\\
    \rule{\textwidth}{0.1cm}\\
    \vspace{0.5cm}
    Implementation Proposal 2020\\
    \vspace{0.5cm}
    \rule{\textwidth}{0.1cm}\\
  }
  \vspace{\stretch{4}}
  {\bfseries \today}
\end{center}

\newpage

\section{Introduction}

For RoboCup 2020, the RoadMap envisions multiple structural changes for the Humanoid League.

The leagues we intend to run are:

\begin{description}[leftmargin=5em,style=nextline]
\item[HSC:] Humanoid Soccer Competition with size classes Kid and Adult
\item[HOC:] Humanoid Open Competition with size classes Kid and Adult
\item[HRD:] Humanoid Research Demonstration
\end{description}

In this document we detail the implementation of our proposal in terms of general tournament procedure, trophies, certificates and implications for the venue.

\section{Humanoid Soccer Competition}
The Humanoid Soccer competition is the new name of the main tournament as it was played in the past years. The major changes foreseen for the league are:

\begin{itemize}
\item From 2020 onwards, \textbf{TeenSize will be merged with KidSize and AdultSize}. The size restrictions of both KidSize and AdultSize will be adjusted respectively. Based on the robot inspection sheets from last year, all TeenSize robots that competed in 2019 would play in KidSize in 2020.
\item The \textbf{Drop-In competition} will be carried out on the first day of the tournament and the results will be used for seeding in the main tournament. Participation in Drop-In only will not be possible anymore.
\item If a size class has more than 16 participating teams,
  it will be divided into division A and division B.
  Division A will be for the 12 highest ranked teams in the Drop-In games.
  The remaining teams (up to 12) will play in division B.
  Similar to the SPL, we intend to run playoff games to allow best
  teams of division B to enter division A.
  The division shall ensure challenging and fun games for all participating
  teams independent of their current level of performance.
  Teams that cannot provide a full team of robots will be merged after the
  initial round of Drop-In games to compete in either division A or division B.
\end{itemize}

\subsection{Number of Teams}
In the following table, we provide the number of teams registered in 2016 (the last RoboCup competition in Europe), the teams that participated last year and our expectation for the 2020 competition:

\begin{table}[h]
  \centering
  \begin{tabular}{l | r | r | r}
    League & 2016 & 2019 & 2020\\
    \hline
    HSC Kid & 17 & 16 & 20-24\\ 
    HSC Teen & 8 & 8 & \\ 
    HSC Adult & 8 & 5 & 5-8\\ 
    \hline
    \textbf{Total} & 33 & 29 & 25-36
  \end{tabular}
  \caption{Number of teams in the Humanoid Soccer Competition.}
\end{table}

\subsection{Trophies and Certificates}
Given the number of teams we foresee in KidSize and AdultSize,
we anticipate that KidSize will be divided into division A and B,
while AdultSize will not be further divided.
Teams from division B will only receive a certificate and not a trophy.
Due to the merge of TeenSize and the reduction of 4 to 5 trophies in this league
(depending on the number of participating teams),
we will in total reduce the number of trophies in the Humanoid Soccer Competition.

\begin{table}[h]
  \centering
  \begin{tabular}{l | r | r}
    League & 2019 & 2020\\
    \hline
    HSC Kid Main tournament & 3 & 3\\ 
    HSC Kid Technical Challenges & 1 & 1\\
    HSC Kid Drop-In & 1 & 1\\
    HSC Teen Main tournament & 2 & \\ 
    HSC Teen Technical Challenges & 1 & \\
    HSC Teen Drop-In & 1 & \\
    HSC Adult Main tournament & 3 & 3\\ 
    HSC Adult Technical Challenges & 1 & 1\\
    HSC Adult Drop-In & 1 & 1\\
    \hline
    \textbf{Total} & 14 & 10
  \end{tabular}
  \caption{Number of trophies in the Humanoid Soccer Competition.}

\end{table}

In addition to the trophies and in compliance with the past two years, we will hand out a Best Humanoid Certificate (unless a new sponsor for a trophy is found) and a Best Referee Certificate.

\subsection{Registration Fees}
According to the decision by the RoboCup federation, the registration fee for new teams will be waived for 2020 again and thus the registration process and fees will remain unchanged for the Humanoid Soccer Competition.

\subsection{Requirements for the Venue}

At RoboCup 2019, we managed to successfully run the competition despite the late
announcement that we will have one field less than expected. We originally requested 5 KidSize and 1 AdultSize field, but were given 4 KidSize and 1 AdultSize field. However, one of the KidSize fields was placed in the lobby under different light conditions than the other fields. While we did ask for a field with natural light conditions, the significant distance between this field and the team area added another level of scheduling constraints. Thus, the organization of the first three days of tournament was heavily impacted by the number of fields available: We had to use 45 minutes time slots instead of the usual one hour time slots, which heavily reduced on-site testing and practice for the teams. Fields were only available for practice before 9am and after 6pm, which many teams reported as a struggle for their tournament success. 

In order to avoid similar issues and ensure a high quality of the tournament,
we need more fields than the number available during RoboCup 2019.
The restructuration of the Humanoid Soccer Competition only add a minimal number
of games due to the playoff between best teams of division A and division B.
We thus ask for 5 KidSize and 2 AdultSize fields for running the Humanoid Soccer
Competition in 2020 (which is one field more than we asked for in 2019 and two
fields more than provided in 2019).

\section{Humanoid Open Competition}

The Humanoid Open Competition is a soccer competition with much reduced laws of the game and with little hardware constraints. While robots still need to show bipedal walking, they may use different sensors (e.g. laser sensors, compass sensors,...) and do not need to follow strict body proportions in comparison to the Humanoid Soccer Competition. However, to ensure fairness of the game, we will sort the robots into the two size classes KidSize ($< 100 cm$) and AdultSize ($> 100 cm$). 

\subsection{Number of Teams}
The laws of the game will ensure that the tournament can also be run with a small number of participating teams in the first years of implementing the league. Thus, we would like to open the league if at least two teams are registering in a size class. Based on initial advertisements of the league, we anticipate the following number of teams registering:

\begin{table}[h]
  \centering
  \begin{tabular}{l | r | r}
    League & 2019 & 2020\\
    \hline
    HOC KidSize & -- & 4-8\\
    HOC AdultSize & -- & 2-4\\ 
    \hline
    \textbf{Total} & -- & 6 - 12
  \end{tabular}
  \caption{Number of teams anticipated in the Humanoid Open Competition.}

\end{table}

\subsection{Trophies and Certificates}
Due to the comparably low number of participating teams foreseen to participate in the first year, we suggest to award only one trophy to the team placed first in KidSize and one to the team placed first in AdultSize. Depending on the number of participating teams, teams placed second and third may receive a certificate.

\subsection{Registration Fees}

For the Humanoid Open Competition,
we envision the registration process and fees to work similar to the Humanoid
Soccer Competition.

\subsection{Requirements for the Venue}
The Humanoid Open Competition will be played on the same fields as the Humanoid
Soccer Competition and if we receive the number of fields we request,
we anticipate that both tournaments can be run in parallel on the same set of
fields because the two tournaments have not the same constraints.
In addition to the fields, we would require 4 additional balls and 4 cardboard or wooden poles to be placed on the fields. 

\section{Humanoid Research Demonstration}
In the Humanoid Research Demonstration, we invite research groups to demonstrate some advancement in either hardware or software that is relevant to humanoid soccer. No full robot is required for the demonstration and we specifically encourage early and ongoing projects to be presented in the demonstration.

\subsection{Number of Teams}

It is particularly difficult to anticipate the number of application we will
receive for the Humanoid Research Demonstration due to the open nature of the
league.
We do believe that our chances of recruiting teams to participate in the
demonstration will be significantly larger once the league has been approved by
the Trustees and an actual call for participation is sent out.
The humanoid league TC will then send personnal calls to specific contacts to a
list of personnal contact in institutes and companies,
it will also kindly ask the Trustees in collaboration with the Technical
Committee to send the CfP out to research teams that are working on robotics
but have not participated in RoboCup so far.
We see the initial call for participation as an important step to start a
conversation with these researchers outside RoboCup and their feedback
regardless of their willingness to participate will be valuable to improve the
league in the following years. 

For the first year, we hope to attract 5 to 10 teams to the Humanoid Research Demonstration.

\subsection{Trophies and Certificates}
The Research Demonstration will not be designed as a competition but as a demonstration. We will thus not award a trophy in this league and only hand out certificates of participation. 

\subsection{Registration Fees}

Since the Research Demonstration is a competition-free event and we will only
require a very limited general space to prepare for the demonstration,
we would propose that the team registration fee is entirely waived for this
league,
regardless of the team's status (new or repeated participation).
However, team members still need to register regularly.
This would allow teams already participating in another RoboCup league to show a
demonstration without additional extra cost and might attract research teams and
companies outside of the RoboCup to be implicated in the league. 

\subsection{Requirements for the Venue}

We anticipate that all research demonstrations can be carried out on the AdultSize field of the competition and that no additional hardware is required for the demonstration. As in previous years, we hope to be provided with a sound system for game comments and screens to display the score to the audience. Both the sound system and the screen would be used for the demonstration.
 
\section{Summary}

\subsection{Number of Teams}
By introducing the two new leagues, we expect to have at least ten more teams / research institutes participating in the RoboCup Humanoid League this year:

\begin{table}[h]
  \centering
  \begin{tabular}{l | r | r}
    League & 2019 & 2020\\
    \hline
    HSC & 29 & 25-32\\ 
    HOC & -- & 6-12\\
    HRD & -- & 5-10\\
    \hline
    \textbf{Total} & 29 & 36-54
  \end{tabular}
  \caption{Overall number of teams in the RoboCup Humanoid League.}

\end{table}


\subsection{Trophies and Certificates}

In 2019, 14 trophies were awarded by the Humanoid League. However, we did calculate with 15 trophies under the assumption that enough teams participate in all three size classes to hand out three trophies in each class. 

As you can see in the following table,
by discontinuing TeenSize in the Humanoid Soccer Competition we decrease the
number of trophies awarded in this league,
which gives us the option to hand out two trophies in the Humanoid Open
Competition and still reduce the overall number of trophies.

\begin{table}[h]
  \centering
  \begin{tabular}{l | r | r}
    League & 2019 (calculated) & 2020\\
    \hline
    HSC & 14 (15) & 10\\ 
    HOC & -- & 2\\ 
    HRD & -- & 0\\
    \hline
    \textbf{Total} & 14 (15) & 12
  \end{tabular}
   \caption{Overall number of trophies in the RoboCup Humanoid League.}

\end{table}

\subsection{Requirements for the Venue}

Opening the new leagues has limited impact on the requirements for the venue.
We expect HOC KidSize to share the fields with HSC KidSize and HOC AdultSize to share fields with HSC AdultSize,
HRD will be carried out on the regular soccer fields and will therefore not require
extra material.


% Time slots for the first 3 days:
% - RR: Round-Robin
% - KO: Knock-out game
% - DI: Drop-In
% - TC: Technical Challenge
% 2019:
% - Kid time slots: 24RR + 12KO + 10DI + 4TC -> 50
% - Teen time slots: 12RR + 6KO + 8DI + 2TC -> 28
% - Adult time slots: 10RR + 2KO + 8DI + 2TC -> 22
% 2020:
% - HSCK time slots (12DivA,12DivB): 16DI + 48RR + 20KO + 6TC -> 90
%   - Round 1 : 24 RR (2* 12, 3 Groups of 4)
%   - Play-offs: 4 KO
%   - Round 2 : 24 RR (2* 12, 3 Groups of 4)
%   - Finals: 16 KO (2*8)
% - HOCK time slots: 5-15DI
% - HSCA time slots: 10RR + 2KO + 8DI + 2TC -> 22
% - HOCA time slots: 4-8
% - HRD  time slots: 4-10
\begin{table}[h]
  \centering
  \begin{tabular}{l|r|r|r|r|}
    & \multicolumn{2}{c}{2019} & \multicolumn{2}{c}{2020}\\
    & Time slots required & Nb fields & Time slots expected & Nb fields\\
    \hline
    Small field: $11m \cdot 8m$ & 78 & 4 & 95-105 & 5\\
    Large field: $16m \cdot 11m$ & 22 & 1 & 30-40 & 2\\
    % 4 persons working table: & 70 & 80-120
  \end{tabular}
  \caption{Time slots and number of fields required to run the RoboCup Humanoid League.}

\end{table}

\end{document}