\documentclass{article}

\usepackage[
    backend=bibtex, 
    natbib=true,
    style=alphabetic,
    citestyle=alphabetic,
    maxcitenames=1,
    maxbibnames=99
]{biblatex}

\usepackage[utf8]{inputenc} 
\usepackage[T1]{fontenc}
\usepackage{enumitem}
\usepackage{lmodern}
\usepackage{graphicx}
\usepackage{tikz}
\usetikzlibrary{calc}
\usetikzlibrary{positioning}
\usepackage{url}
\usepackage{rotating}
\usepackage{graphicx}            % For pdf/bitmapped graphics files
\usepackage[margin=3cm]{geometry}
\graphicspath{{images/}}
\DeclareGraphicsExtensions{.pdf,.png,.jpg,.jpeg}

\usepackage{hyperref}
\usepackage{todonotes}

\begin{document}


\begin{center}
  {\Huge \bfseries
    RoboCup Soccer: Humanoid League
    \vspace{\stretch{1}}\\
    \rule{\textwidth}{0.1cm}\\
    \vspace{0.5cm}
    Implementation plan proposal 2020\\
    \vspace{0.5cm}
    \rule{\textwidth}{0.1cm}\\
  }
  \vspace{\stretch{4}}
  {\bfseries \today}
\end{center}

\newpage

\section{Introduction}

For RoboCup 2020, we have multiple structural changes for the Humanoid League.
The Teen Size class will merge within both Kid and Adult and several new leagues
will be introduced.

The leagues we intend to run are:

\begin{description}[leftmargin=5em,style=nextline]
\item[HSCK:] Humanoid Soccer Competition Kid, (merge of Humanoid Kid Size and
  Humanoid Teen Size, will likely result in two divisions)
\item[HSCA:] Humanoid Soccer Competition Adult, (merge of Humanoid Adult Size
  and Humanoid Teen Size)
\item[HOCK:] {Humanoid Open Competition Kid}
\item[HOCA:] {Humanoid Open Competition Adult}
\item[HRD:] {Humanoid Research Demonstration}
\end{description}

In this document we detail the implementation of our proposal in terms of
trophies, certificates and impact on requirements for the venue.

\section{Number of teams per league}

The number of teams who registered for RoboCup 2019 and our expectations for
RoboCup 2020 are presented in Table~\ref{tab:nb_teams}.
It is particularly difficult to anticipate the number of application we will
have for the Humanoid Open Competitions and the Humanoid Research Demonstration.
However, we can mention the following elements:
\begin{itemize}
\item HOCK will not be opened if there are less than four teams registering.
\item For HOCA, we would like to be able to open it even if there are only 2
  teams registering.
  The number of teams participating to the RoboCup with adult size robots is
  currently too low to afford to lose candidates.
\item Since HRD is not a competitive event, but a research exhibition,
  it does not require a minimal amount of teams to be meaningful.
  \todo{If we add judging, this would not be valid}
\end{itemize}

\begin{table}[h]
  \centering
  \caption{\label{tab:nb_teams}Number of teams in 2019 and expected
    participation for 2020}
  \begin{tabular}{l | r | r}
    League & 2019 & 2020\\
    \hline
    HSCK: & 16 & 20-30\\ 
    HSCT: & 8 & \\ 
    HSCA: & 5 & 5-8\\ 
    HOCK: &  & 4-8\\
    HOCA: &  & 2-4\\ 
    HRD: &  & 5-10\\
    \hline
    \textbf{Total} & 29 & 36-60
  \end{tabular}
\end{table}


\section{Trophies and certificates}

The evolution of the number of trophies is presented in
Table~\ref{tab:trophies}\todo{Confirm 2019 numbers}.
The slight increase in the number of trophies is conditioned to having enough
teams registering for all the leagues, which would lead to a global increase of
the number of teams.
Therefore, the average number of trophies requested per team tend to decrease
with our proposal.
Since the \emph{Best Humanoid Award} is conditioned to obtaining a new sponsor,
we do not consider it here.
\begin{table}[h]
  \centering
  \caption{\label{tab:trophies}Trophies evolution}
  \begin{tabular}{l | r | r}
    League & 2019 & 2020\\
    \hline
    HSCK: Main tournament & 3 & 6\\ 
    HSCK: Technical Challenges & 1 & 1\\
    HSCK: Drop-In & 1 & 1\\
    HSCT: Main tournament & 2 & \\ 
    HSCT: Technical Challenges & 1 & \\
    HSCT: Drop-In & 1 & \\
    HSCA: Main tournament & 3 & 3\\ 
    HSCA: Technical Challenges & 1 & 1\\
    HSCA: Drop-In & 1 & 1\\
    HOCK: &  & 1\\
    HOCA: &  & 1\\ 
    HRD: &  & 0\\
    \hline
    \textbf{Total} & 14 & 15
  \end{tabular}
\end{table}


\todo{Certificate List}

\section{Registration fees}

For the registration fee, we propose different approaches depending on the
league:

\begin{description}
\item[HSCK and HSCA:] We keep the existing scheme of having free team
  registration for new teams while all members have to register.
\item[HOCK and HOCA:] For RoboCup 2020, we propose that to have free team
  registration since both are new leagues.
  Then, we can keep the same scheme with free team registration for new teams.
  In both cases, members still have to register.
\item[HRD:] \todo{Content depending on award/not}
\end{description}

\section{Requirements for the venue}

Opening the new leagues has limited impact on the requirements for the venue.
We expect HOCK to share the fields with HSCK and HOCA to share fields with HSCA,
HRD will also happen on the regular soccer fields and will therefore not require
extra materials.
As is shown in table~\ref{tab:nb_teams}, we expect less teams in the new
leagues,
therefore there should be a reasonable increase in terms of number of
participants.

\todo{Compare fields requirements + talk about space usage last year}

\end{document}
