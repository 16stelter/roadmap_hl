\documentclass{article}

\usepackage{url}

\begin{document}

\section{Vote Results (brief)}

\begin{itemize} 
   \item Solved issues
   \begin{itemize} 
       \item Open humanoid league
       \begin{itemize} 
           \item With 2 differents leagues
   \end{itemize}
       \item Creation of division A/B (given there are enough teams)
       \item Division B rules will evolve at a slower pace
       \item Leagues are discontinued only by lack of participants
       \item Teams can participate to technical challenges even if they do not participate to major leagues
       \item We will have technical challenges until 2050
       \item We will have Open Humanoid demo
       \begin{itemize} 
           \item With proposed themes regularly changing
           \item No need to have a full robot
           \item There will be no award
   \end{itemize}
       \item Teams will not be forced to build joint team (it will be implicit by requiring more robots)
       \item Hardware for 2050
       \begin{itemize} 
           \item Safety constraints
           \item Human-like
           \begin{itemize} 
               \item Passive perception
               \item Field of view limitation
               \item Wavelength spectrum
       \end{itemize}
           \item Not limited
           \begin{itemize} 
               \item Resolution
               \item Frequency
               \item Computational power
               \item Computing power restriction
               \item Inertial sensors
               \item WiFi (authorized)
   \end{itemize}
   \end{itemize}
       \item We should focus toward playing vs humans ASAP
       \item We should reduce hardware constraints for the main leagues in the short-term
       \item Rules should be updated every 3 years
       \item Team representative should be able to vote on changes, but after the competition
\end{itemize}
   \item TBD (mixed opinions)
   \begin{itemize} 
       \item Reduced hardware constraints for technical challenges?
       \item Which number of technical challenges? 3 or 4?
       \item Drop-In games? Should we abandon? Which kind of Drop-In
       \item Torque/speed limitations for joints?
   \end{itemize}

\end{itemize}
\section{Vote Results (detailed)}
\subsection{Section 1: Sub-leagues}

\begin{itemize} 
   \item Open Humanoid League accepted ( 5/1)
   \item One or two leagues for "Open Humanoid" (uncertain 3/3)
   \begin{itemize} 
       \item Trustees: Mitigated (2 for 1 league, 2 for 2 leagues, 2 abstains)
       \item Teams: 2 leagues (9 for 2 leagues, 6 abstains, 2 for one league)
\end{itemize}
   \item Creation of Division A/Division B (uncertain 4/2)
   \begin{itemize} 
       \item Trustees: Approved (5 for, 1 abstain)
           \begin{itemize} 
               \item Resoluti
   \end{itemize}
       \item Teams: Approved (11 for, 5 against, 1 abstain)
\end{itemize}
   \item Different rules for Division B (uncertain 3/3)
   \begin{itemize} 
       \item Trustees: Mitigated (3 for, 3 against)
       \item Teams: Approved (10 for, 5 against, 2 abstains)
\end{itemize}
   \item When are leagues discontinued: Lack of participants (5/1)
   \item Q6+Q7: Not applicable
\end{itemize}
\subsection{Section 2: Technical challenges}

\begin{itemize} 
   \item Teams can participate to TC even if they do not participate to Major Leagues ( 5/1)
   \item \textbf{\textit{Reduce hardware constraints for TC (uncertain 3/3)}}
   \begin{itemize} 
       \item \textbf{\textit{Trustees: for (4 for, 2 against)}}
       \item \textbf{\textit{Teams: against (12 against, 5 for)}}
\end{itemize}
   \item \textbf{\textit{Increase number of TC (uncertain 3/3)}}
   \begin{itemize} 
       \item \textbf{\textit{Trustees: Less than 4 (5:<4, 1:4)}}
       \item \textbf{\textit{Teams: 4 or less (9:4, 6:<4,2:abstain)}}
\end{itemize}
   \item We shall have TC until 2050 (5/1)
\end{itemize}
\subsection{Section 3: Demo Events}

\begin{itemize} 
   \item Open Humanoid Demo (uncertain 3/3)
   \begin{itemize} 
       \item Trustees: for (5 for, 1 abstain)
       \item Teams: for (14 for, 2 against, 1 abstain)
\end{itemize}
   \item Proposed themes accepted (6/0)
   \item Themes changing regularly (5/1)
   \item No need to have a full humanoid robot (5/1)
   \item Is an award provided for the demo (uncertain 2/4)
   \begin{itemize} 
       \item Trustees: against (5: against, 1 for)
       \item Teams: Not asked
   \end{itemize}

\end{itemize}
\subsection{Section 4: Drop-In games}

\begin{itemize} 
   \item \textbf{\textit{Drop-In only/Drop-In removal (uncertain 3/3)}}
   \begin{itemize} 
       \item \textbf{\textit{Trustees: Drop-in Only (2 Drop-in only, 1 abandon DI, 3 abstains)}}
       \item \textbf{\textit{Teams: Abandon DI (7 Abandon DI, 6 abstain, 4 DI only) <- comments requiring other options}}
\end{itemize}
   \item \textbf{\textit{Fixed team / Changing teams (uncertain 3/3)}}
   \begin{itemize} 
       \item \textbf{\textit{Trustees: Fixed teams (4: Fixed teams, 2: Abstains)}}
       \item \textbf{\textit{Teams: Changing teams (9: changing, 5: fixed, 3: abstains)}}
\end{itemize}
   \item Force building joint teams (uncertain 2/4)
   \begin{itemize} 
       \item Trustees: Mitigated (2 for, 2 against, 2 abstains)
       \item Teams: Against (10 against, 5 abstains, 2 for)
\end{itemize}
   \item No limit to the number of robots provided per team (5/1)
\end{itemize}
\subsection{Section 5: Hardware of Humanoid robots}

\begin{itemize} 
   \item Decided
   \begin{itemize} 
       \item No frequency limit (5/1)
       \item No resolution limit (5/1)
       \item No computational power limits (6/0)
       \item No constraints on location of the 'brain' (5/1)
\end{itemize}
   \item Uncertain
   \begin{itemize} 
       \item Passive perception (3/3)
       \begin{itemize} 
           \item Trustees: Yes (4: for, 2 against)
           \item Teams: Yes (14: for, 3 against)
   \end{itemize}
       \item Field of view limit (3/3)
       \begin{itemize} 
           \item Trustees: Yes (5: for, 1: against)
           \item Teams: Yes (14: for, 3: against)
   \end{itemize}
       \item Wavelength spectrum limit (3/3)
       \begin{itemize} 
           \item Trustees: Mitigated (2: for, 2: against, 2:abstain)
           \item Teams: Yes (9: for, 6: against, 2: abstain)
   \end{itemize}
       \item \textbf{\textit{Limit to torque/speed for joints (3/3)}}
       \begin{itemize} 
           \item \textbf{\textit{Trustees: Yes (3: for, 2: against, 1: abstain)}}
           \item \textbf{\textit{Teams: No (11: against, 6: for)}}
   \end{itemize}
       \item Inertial sensors constraints (3/3)
       \begin{itemize} 
           \item Trustees: No (5: against, 1: for)
           \item Teams: Mitigated (8: against, 8: for, 1: abstain)
   \end{itemize}
       \item WiFi communication (3/3)
       \begin{itemize} 
           \item Trustees: Authorized (4: for, 2: against)
           \item Teams: Authorized (11: for, 6: against)
\end{itemize}
\end{itemize}
   \item New: Should hardware constraints be imposed to ensure human safety
   \begin{itemize} 
       \item Trustees: Yes (3: for, 1:against)
   \end{itemize}

\end{itemize}
\subsection{Section 6: Reaching the 2050 goal}

\begin{itemize} 
   \item Focus on playing against humans ASAP (5/1)
   \item Temporary reduction of hardware constraints (3/3)
   \begin{itemize} 
       \item Trustees: Yes (4: for, 2: abstain)
       \item Teams: Yes (9: for, 8: abstain)
   \end{itemize}

\end{itemize}
\subsection{Section 7: Rule book update <- Not asking this to trustees}

\begin{itemize} 
   \item Frequency update
   \begin{itemize} 
       \item Less frequently (4/2)
       \begin{itemize} 
           \item Teams: Less frequently (13: for, 4: against)
   \end{itemize}
       \item Every 3 years (4/2)
       \begin{itemize} 
           \item Teams: 3 years (8: three years, 6: two years, 2: four years, 1: abstain)
\end{itemize}
\end{itemize}
   \item Team Leaders: giving opinions or voting on rule changes? (uncertain 3/3)
   \begin{itemize} 
       \item Teams: Voting (11: voting, 5: give opinion, 1: abstain)
\end{itemize}
   \item Vote should be performed later during the year (5/1)
   \begin{itemize} 
       \item Teams: later during the year (9: later, 8: during tournament)
   \end{itemize}

\end{itemize}
\subsection{Section 8: What did we miss?}

\begin{itemize} 
   \item What Technical Challenges shall we have in the future? 5 responses
   \begin{itemize} 
       \item TC
       \begin{itemize} 
           \item Foot-step planning
           \item Challenge based on themes: "Acting in dynamic situations (kick from moving ball, volley kick, aerial kick (robot out of ground during kick)" "Robustness (Push-recovery and fall impact minimization, while walking and then running)" "Dexterity/ball manipulation (Ground dribbling, Juggling, etc...)" "Power increase (high kick, long distance kick, ...)" "Accuracy (kicking in a reduced size goal, small window,  etc...)" "Locomotion (Walking for a defined trajectory, running, tackling)"
           \item Running, Change high-kick challenge so that it has to be performed from the penalty mark and then later so that the robot needs to hit certain marked spots in the goal (upper left corner),  Penalty kicks against a human goal keeper
           \item New technical challenges focusing on the robots' upper bodies capabilities 
           \item Running, Long Jump, Stand-up in AdultSize
   \end{itemize}
       \item Trustees
       \begin{itemize} 
           \item See research question
           \item Fast dribbling exercises (robot dribbles in zig-zag around a line of poles)
           \item Check the new EU project EUROBENCH. They are proposing several challenges that could lead to benchmarks for humanoid robots. Those could be inspiring for RoboCup HL technical challenges
           \item See research question
   \end{itemize}
       \item Teams
       \begin{itemize} 
           \item running
           \item Sprint
           \item Walking on uneven terrain
           \item Robots challenge to pass a ball to different directions by on-line planning the local motion.
           \item Natural light test. Whistle detection test (detect the location of a whistle when it is blown). Goalie (successfully save a goal). Uneven surface test (walk across an uneven surface without falling over, how quickly can you do it). Balance test (robot walking, hit with a bottle from an oblique angle)
           \item swerve around sticks
           \item Goalie Technical Chalenges
           \item Action Recognition Challenge
           \item Dribbling through an obstacle course, like a training drill
           \item Heading
           \item Penalty Kick (Demonstrates the overall responsiveness of the system), Jump the highest
\end{itemize}
\end{itemize}
   \item What research questions do you believe the development of the league shall be driven by?
   \begin{itemize} 
       \item TC: 6 responses
       \begin{itemize} 
           \item How to improve robot collaboration
           \item Robustness in robotics, Decision-Making under uncertainty, Perception in dynamic environments, Dynamic Locomotion, Soft/Safe Robotics
           \item Falling robots, robustness and durability while still being humanoid, human-robot interaction safety, dynamic and precise movement, multi-agent communication, natural language communication under extreme environments (background noise and long distances) 
           \item NAN
           \item Increasing walking speed, being able to run, being able to fall without significant damage
           \item Running robots
   \end{itemize}
       \item Trustees:
       \begin{itemize} 
           \item Soft Robotics, on-the-fly adaptation, hierarchical walk control
           \item Achieve human-like movements (running, walking, shooting, ..)
           \item Focus more on relevant research questions for humanoid robots in general, and not so much on constraints towards making them similar to humans. This may come later, far away from now, when the robots will be closer to be able to play against humans.
           \item There are so many issues to be cited. Among them, artificial muscles, whole body tactile sensation, brain architecture connecting conscious and unconscious processing. 
   \end{itemize}
       \item Teams:
       \begin{itemize} 
           \item robot fast self-positioning and recognize enemy/friend
           \item I think the biggest advantage in RoboCup is having a test bed for robustness in "real world" scenario. Research in that area is naturally one of the main possibilities. Although it is not yet very highly supported in research proposals, I would assume that this change given the overall intention of having robots in the wild soon. Robustness involves not only technical robustness, but also algorithmical robustness, e.g., vision and locomotion. 
           \item The motion generation of special actions which only exist in soccer for multi-bot cooperation, such as passing in a specific direction.
           \item How can we integrate all necessary components in a cost efficient humanoid robot, while maintaining robustness?
           \item Robust and generic computer vision. Bipedal motion on uneven surfaces. Audio-based communication.
           \item Direct Drive Motor(Hardware), Perception
           \item bipedal control
           \item complex human-like bipedal motions localization (using the human-like sensors only), distributed AI and its challenges
\end{itemize}
\end{itemize}
   \item Are there any other main guiding questions we should vote on?2 responses
   \begin{itemize} 
       \item Should we force teams to open source their software/hardware designs?
       \item Why just focusing on soccer for humanoid robots researches?
\end{itemize}
   \item Missing points?
   \begin{itemize} 
       \item Trustees:
       \begin{itemize} 
           \item I think most points are well presented, as far as I am concerned
           \item Nothing
   \end{itemize}
       \item Teams
       \begin{itemize} 
           \item There was no possibility to say that Drop in games could be left as they are while still having regular games
           \item All teams need to become open source (mandatory code release for any team that achieves a podium position, penalty for next years qualification if code release has not happened and open source status is not confirmed - this should help all new teams who are joining the leagues). Encourage teams to open source their custom developed tools/datasets/etc. 
           \item Approaches to make the event more interesting to public: visualize robots decisions for spectators, comments on the decisions, movments etc. Like in the FIA-Event Formular 1
\end{itemize}
\end{itemize}
   \item Open suggestions for Roadmap
   \begin{itemize} 
       \item Trustees
       \begin{itemize} 
           \item Dito (see also our discussion at IROS).
           \item Multiple comments:
           \begin{itemize} 
               \item Regarding 1.1), I don't think it's necessary to decide the leagues in advance.  I suggest making an open call, and then based on the respondents, decide if they can fit in one league, or if they need to be broken up based on capabilities.  In other words, don't place any limitations in the open call.  Be adaptive based on who responds.
               \item Regarding 1.2), I suggest patterning off of the SPL.  Have an upper division and a lower division, with winners of the lower division being given an opportunity to challenge teams in the upper division.
               \item Regarding 2.2), I don't think there should be any constraint on the number of challenges.  It could be more or from year to year less depending on what they are and levels of interest/participation.
               \item Regarding 5.6), eventually hardware constraints will definitely need to be placed for human safety.  But I don't think that needs to happen until the time when robots are ready to play people. 
               \item Regarding overall constraints, I continue to stand by my suggestions in this article:    \url{http://www.cs.utexas.edu/~pstone/Papers/bib2html/b2hd-PHILOSOC10.html}
       \end{itemize}
           \item Technical challenge demonstration should be invited-base. That is, TC selected the research group or companies that show excellent performance in some topic  which TC think important. To do that, TC may request the money for invitation. Trustees may help to contact such organizations.
   \end{itemize}
       \item Teams
       \begin{itemize} 
           \item We would like to have an early release date of the rulebook, which is known beforehand. We also believe it is important to limit rule change frequency to a minimum.
           \item While we voted to abandon drop-in, what we really wanted to vote for was restructuring drop-in to make it run more efficiently (enforce a standardised communications protocol, better limit on the number of games a team will play in a day (including normal tournament games)). Bring back best humanoid (we can supply 3D printed trophy if needed).
           \item Integration of SimLeague with Humanoid League in the long term
       \end{itemize}

   \end{itemize}

\end{itemize}


\section{Summary}
\subsection{Attendance}

\begin{itemize} 
   \item ~25 persons during invited talks
   \item ~20 persons for discussions
   \item 2 were present at RoboCup2002
   \item Most of people were present at Montreal
   \item Only Minoru Asada listened to Raibert's keynote at IROS
\end{itemize}
\subsection{Invited Talks}

\begin{itemize} 
   \item Suggested aims:
   \begin{itemize} 
       \item Attract new technologies and new ideas
       \item Take into account Human Robot Interaction
       \item Benchmark internal system abilities (Why some team perfom better than other)
       \item Improve fluidity of games
   \end{itemize}

\end{itemize}
\subsection{Discussion summary}

\begin{itemize} 
   \item Roadmap content
   \begin{itemize} 
       \item Request to discuss league futur
       \item Technical challenges require update (More manipulation skills)
       \item Including missing rules in RoadMap (e.g. offside)
       \item Effectively applying rules already active (free kick)
       \item More encouragement for collaboration between teams
       \item Request for availability of rulebook sooner (more time to prepare to rule changes)
       \item Take into account financial aspects for the teams when building the roadmap
       \item Ensure changes which allows to make the league more popular (all groups)
\end{itemize}
   \item Improve cooperation
   \begin{itemize} 
       \item More local events (not enough team is still a problem)
       \item Promote long-term cooperation between teams
       \item Mentorship for new teams
\end{itemize}
   \item Attract new teams
   \item Communication
   \begin{itemize} 
       \item For everyone
       \begin{itemize} 
           \item Display more visual information during the game
           \item Collect data on the robot skillsteams 
   \end{itemize}
       \item With researcher
       \begin{itemize} 
           \item More advertising in robotics conference and workshops
           \item Make history of improvement more accessible
   \end{itemize}
       \item With public
       \begin{itemize} 
           \item Improve the gameplay quality
           \item Improve communication
   \end{itemize}
       \item More media coverage to get more investors
\end{itemize}
   \item RoboCup level remarks
   \begin{itemize} 
       \item Financial support for teams who can't pay travel
       \item Promote support for long-term project 
   \end{itemize}
\end{itemize}

\section{Discussion}

\begin{enumerate} 
   \item Content of Roadmap
   \begin{enumerate} 
       \item What can be elements of a HL roadmap?
       \begin{itemize} 
           \item \underline{Planning which sizes to continue  (do we want kid size in 2050?)}
           \item What is our final goal in detail (WiFi, GameController, listening to the referee)
           \item Ethical behaviour of robot (how "bad" can a robot behave)
           \item \underline{Missing rules (offside)}
           \item \underline{Technical challenges}
           \begin{itemize} 
               \item \underline{Running but not in 2020}
               \item \underline{Jumping $\rightarrow$ switch to long jump}
               \item \underline{Ball handling skills (dribbling, multi agent, juggling)}
               \item \underline{Grasping the ball (GoalKeeper throw-in)}
   \end{itemize}
   \end{itemize}
       \item How can control mechanisms of a HL roadmap look?
       \begin{itemize} 
           \item Control the number of teams (avoid dying)
           \item \underline{Encourage collaboration of teams}
           \item Separate Open-source project for every league
           \item Creating the collection of frontiers tasks
           \item \underline{Rules should be clear for the teams longer in advance}
           \item Committes for the different tasks of the roadmap
   \end{itemize}
       \item How should a budget / venue aware HL roadmap look?
       \begin{itemize} 
           \item Increase the field size (artificial ground more expensive)
           \item \underline{Scholarship for teams that can not pay the travel (support these teams)}
           \item Set a maximum money that the teams can be asked to invest for the next year
           \item More robots $\rightarrow$ more costs (material, people, etc...)
           \item \underline{More/better media for the RoboCup $\rightarrow$ get some more investors}
\end{itemize}
\end{enumerate}
   \item Scientific aspects and progress in Roadmap
   \begin{enumerate} 
       \item Which scientific and technical aspects should be reflected in a HL roadmap?
       \begin{itemize} 
           \item Field size <$\rightarrow$ Walking speed + Kick distance/accuracy
           \begin{itemize} 
               \item Detection: Player, Opponent, Goal, Ball, tracking
               \item Robot, human$\rightarrow$ robot, human communication
               \item Self localization and dynamic environment mapping
       \end{itemize}
           \item Game duration
           \begin{itemize} 
               \item Power efficiency
               \item Robustness (DMUP)
               \item Falling, pushing
       \end{itemize}
           \item Number of players (multi-agent)
           \begin{itemize} 
               \item All previously mentioned points
               \item Cooperation
               \item Coaching
               \item Human-help
               \item High-level
   \end{itemize}
   \end{itemize}
       \item Which methods to assess progress should be implemented in a HL roadmap?
       \begin{itemize} 
           \item From logs, deduce statistics
           \begin{itemize} 
               \item In game walking speed: fixed time
               \item Kicking distance: ball distance, height
               \item Localisation quality: Application level (quantitative evaluation) (shared map)
               \item Shots on target/Shots: accuracy
               \item Robustness: untouched falls/errors, untouched pickup/services
               \item Game stats: Ball possession, goals scored, shots on goal, distance traveled, ...
   \end{itemize}
   \end{itemize}
       \item Which elements of a HL roadmap are relevant for the 2050 game?
       \begin{itemize} 
           \item All elements mentioned previously
           \item Increase popularity of RoboCup
           \begin{itemize} 
               \item New teams/Press/Media
       \end{itemize}
           \item Send CFP to other mailing lists
\end{itemize}
\end{enumerate}
   \item Roadmap in community
   \begin{enumerate} 
       \item How could a HL roadmap foster cooperation by participants?
       \begin{itemize} 
           \item Current attempt: Drop-In games
           \begin{itemize} 
               \item Good for practicing, but no collaboration between teams yet
               \item Reaching collaboration in DI games, does not mean there is scientific cooperation
       \end{itemize}
           \item Sharing source code:
           \begin{itemize} 
               \item Comment from @Work: some teams share only partially and without documentation $\rightarrow$ not so useful
       \end{itemize}
           \item Local events (GermanOpen, RoHOW etc):
           \begin{itemize} 
               \item Helping, but not enough teams in some areas
       \end{itemize}
           \item Pushing toward large collaboration projects (a few weeks of visiting student is not enough)
           \item Incentive for cooperation required (forcing a common protocol ?)
           \item Mentorship could help new teams
   \end{itemize}
       \item Which methods of involving new participants can be foreseen in a HL roadmap?
       \begin{itemize} 
           \item Making games more interesting (for public and researchers)
           \begin{itemize} 
               \item Displaying Belief of robots on screen
               \item Gather videos from team to show between games
       \end{itemize}
           \item Make data footage of games (movies + belief of robots)
           \item Reducing development required to enter
           \begin{itemize} 
               \item Improve software/hardware sharing (mandatory?)
       \end{itemize}
           \item Reducing hardware cost
           \begin{itemize} 
               \item Drop-in/Joint teams helps to reduce the number of robots required
               \item Start/encourage hardware projects relevant to the league
   \end{itemize}
   \end{itemize}
       \item How should a HL roadmap be introduced to the community?
       \begin{itemize} 
           \item Presentation
           \begin{itemize} 
               \item Abstract visualization (critical path method)
               \item Motivations behind the choices
       \end{itemize}
           \item Communication
           \begin{itemize} 
               \item Keynotes about it at workshops
               \item Use it as a bases to present work (identify what topic is concerned)
       \end{itemize}
           \item Start planning roadmap updates
           \begin{itemize} 
               \item Add an expiration date on the roadmap
               \item Review status every 1 or 2 years
       \end{itemize}
           \item Keep an history of what has been achieved/rule changes
           \begin{itemize} 
               \item Note: already a paper by Soroush and Jacky, should be next to the Roadmap on the website
       \end{itemize}
           \item Benchmarking of system abilities
           \begin{itemize} 
               \item Helps for justifying change
               \item Helps to understand/present the evolution
       \end{itemize}
           \item Taking footage when milestones are reached
           \begin{itemize} 
               \item Video are useful to have a trace of what was achieved
           \end{itemize}

       \end{itemize}

   \end{enumerate}

\end{enumerate}

\end{document}
