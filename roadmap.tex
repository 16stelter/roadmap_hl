\documentclass{article}

\usepackage[utf8]{inputenc} 
\usepackage[T1]{fontenc}
\usepackage{lmodern}
\usepackage{graphicx}
\usepackage{tikz}
\usetikzlibrary{calc}
\usetikzlibrary{positioning}
\usepackage{url}
\usepackage{rotating}

\begin{document}

\section{Introduction}

% Document: proposal to replace 2014 version: main issues identified
% - Difficult to plan long-term evolution in research -> guided by breakthrough
% - Example of kid-size
This document is a proposal to replace the roadmap proposed in 2014 for the
RoboCup humanoid
league~\footnote{\url{http://www.robocuphumanoid.org/wp-content/uploads/HumanoidLeagueProposedRoadmap.pdf}}. Criticism
have emerged with respect to establishing a long-term schedule for evolution of
the rules in a research context where evolution is mainly guided by breakthrough
rather than by linear development. Some of the changes planned revealed to be
difficult to implement: the removal of the KidSize initially planned for 2020
seems now unrealistic since this league still received more applications than
TeenSize and AdultSize together in 2019. Finally, the relationship between
missing skills and research topics was not explicited.

% Aim of this document + events to gather opinions
In order to avoid facing this issues, a different approach is used. Rather than
establishing a long-term schedule, the document describes the targeted challenge
for the league, the steps required to tackle it. Upcoming changes in the next 5
years are described in more detail to allow teams to plan their developments. In
order to produce a roadmap satisfying for both the trustees and the league, a
workshop was run at IROS2018 and polls have been proposed to both of them. While
a consensus was obtained on most of the questions, this process has also shown
that for some propositions it is difficult to reach an agreement.

\subsection{Goals of the roadmap}

\subsection{Which path to follow?}
% First focus toward improving gameplay

\subsection{Structure of the document}

The structure of the documents is as
follows. Section~\ref{sec:ScientificProblematics} introduces the scientific
challenges related to the league and their importance in the games. The major
upcoming changes to the rules are presented in Section~\ref{sec:ShortTerm}. The
global evolution of the leagues is discussed in Section~\ref{sec:LongTerm}
through an event-triggered Roadmap and the history of changes which occured
since the creation of the league.

\section{\label{sec:ScientificProblematics}Scientific problematics}

\subsection{Motion control for humanoid robots}

% Very brief history reminder
The first humanoid robot to be built is the WABOT-1 which was completed in
1972. While WABOT-1 was already able to walk, controlling humanoid robots remain
a challenging problem today. From designing dynamic walking gait to running and
real-time adaptation of multi-objective motions, motion control of humanoid
robots is an active research topic.

% Focus of RoboCup : introduction
While bipedal locomotion is one of the main interests of the RoboCup humanoid
league, multiple dynamic motions are required such as standing-up and kicking
the ball. Given the adversarial context of the RoboCup, one of the key elements
is also the transition between different motions in order to reduce the time
required to achieve complex tasks. Finally, one the major concern from a
mechanical point of view are achieving satisfying performances while using
low-cost hardware and ensuring the robustness of the hardware despite the fact
that the robots are falling.

\subsubsection{Bipedal locomotion}

% Historically:
% ZMP -> Developed in 1970, still quite used today ``
% Walking, walking on uneven terrain, running

% RoboCup specificities:
% - controlable walk (

\subsubsection{Robustness and reduction of hardware cost}
% - Robots walking all game long (usually, only for demo)
% - Low-cost architecture (risk of damaging the robots, adult-size excepted)

\subsubsection{Motion learning}
% Several specific motions required (standing-up)

\subsubsection{Transition between motions}



\subsection{Embedded perception}

\subsection{Decision-making}

\subsection{Interaction with humans}

\subsection{Cost efficient hardware}

\section{\label{sec:ShortTerm}Short-term: upcoming changes}

\subsection{2019: ...}

Description of the voted changes:
\begin{itemize}
\item Adult Size: field size increase
\item Throw-in, corner kicks, goal kicks
\end{itemize}

\subsection{2020: ...}

TODO: 
\begin{itemize}
\item Introduce Open Humanoid Leagues
\item Introduce Demo Events
\item Updates of technical challenges?
\end{itemize}

\section{\label{sec:LongTerm}Long-term: event triggered roadmap}
\begin{sidewaysfigure}
% This document contains a tentative of dependency diagram for the Humanoid League roadmap

\tikzstyle{element}=[rectangle, draw=black, rounded corners, text centered,
  anchor=north, text width=2cm, minimum height= 1cm]
\tikzstyle{transitionNote}=[rectangle, right, rounded corners, %text left,
  text width=3cm, minimum height= 1cm]
\tikzstyle{dependency} = [->, thick]

\begin{tikzpicture}[node distance=1cm, every node/.style={scale=0.6}]
  % Locomotion
  \node (Walking) [element] {Walking};
  \node (Running) [element, below=of Walking] {Running};
  \draw [dependency] (Walking) -- (Running);
  % Falling column
  % TODO: Still missing: Diving (goalie)
  \node (StandingUp) [element,right=of Walking] {Falling and standing up};
  % TODO: missing ball manipulation (hands + foot)
  % Kicking (add the transition part (kick from walking / kick from running))
  % Add the high kick?
  \node (StaticKick) [element,right=of StandingUp] {Static Kick};
  \node (RollingKick) [element, below=of StaticKick] {Kick from rolling ball};
  \node (FlyingKick) [element, below=of RollingKick] {Kick from flying ball};
  \draw [dependency] (StaticKick) -- (RollingKick);
  \draw [dependency] (RollingKick) -- (FlyingKick);
  % Perception
  \node (BallDetection) [element,right=of StaticKick] {Ball detection};
  \node (RollingBall) [element,below=of BallDetection] {Rolling ball tracking};
  \node (FlyingBall) [element,below=of RollingBall] {Flying ball tracking};
  \draw [dependency] (BallDetection) -- (RollingBall);
  \draw [dependency] (RollingBall) -- (FlyingBall);
  % Environment information
  \node (SidePoles) [element,right=of BallDetection] {Side Poles};
  \node (ColorCoded) [element,below=of SidePoles] {Colored objects};
  \node (HumanLike) [element,below=of ColorCoded] {No colors};
  \draw [dependency] (SidePoles) edge node [transitionNote] {2012: Side poles\\2013: Yellow goals} (ColorCoded) ;
  \draw [dependency] (ColorCoded) edge node [transitionNote] {2015:\\White goals\\White ball} (HumanLike);
  % Ground type (Carpet, Turf, outdoor ground)
  \node (Carpet) [element,right=of SidePoles] {Carpet};
  \node (Turf) [element,below=of Carpet] {Artificial turf};
  \node (RealGrass) [element,below=of Turf] {Real grass};
  \draw [dependency] (Carpet) edge node [transitionNote] {2015: 30mm turf} (Turf);
  \draw [dependency] (Turf) -- (RealGrass);
  
  % Nb players
  \node (1player) [element,right=of Carpet] {1 vs 1};
  \node (2players) [element,below=of 1player] {2 vs 2};
  \node (3players) [element,below=of 2players] {3 vs 3};
  \node (4players) [element,below=of 3players] {4 vs 4};
  \node (6players) [element,below=of 4players] {6 vs 6};
  \node (8players) [element,below=of 6players] {8 vs 8};
  \node (11players) [element,below=of 8players] {11 vs 11};
  \draw [dependency] (1player)  -- (2players);
  \draw [dependency] (2players) -- (3players);
  \draw [dependency] (3players) -- (4players);
  \draw [dependency] (4players) -- (6players);
  \draw [dependency] (6players) -- (8players);
  \draw [dependency] (8players) -- (11players);
  

  % Cross category dependencies
  \draw [dependency] (RollingBall) -- (RollingKick);
  \draw [dependency] (FlyingBall) -- (FlyingKick);
\end{tikzpicture}

%TODO: add removal of magnetometer

\end{sidewaysfigure}

\section{Evolution of the document}

\begin{itemize}
\item Status of the 'long-term' part updated yearly to assess progress
\item Status of the 'short-term' part updated every year or every two years
\item Scientific problematics: updated every 5 years
\end{itemize}

%\section{Past evolutions}


\end{document}
