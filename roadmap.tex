\documentclass{article}

\usepackage[
    backend=bibtex, 
    natbib=true,
    style=alphabetic,
    citestyle=alphabetic,
    maxcitenames=1,
    maxbibnames=99
]{biblatex}

\usepackage[utf8]{inputenc} 
\usepackage[T1]{fontenc}
\usepackage{lmodern}
\usepackage{graphicx}
\usepackage{tikz}
\usetikzlibrary{calc}
\usetikzlibrary{positioning}
\usepackage{url}
\usepackage{rotating}

\addbibresource{roadmap.bib}

\begin{document}

\section{Introduction}

% Document: proposal to replace 2014 version: main issues identified
% - Difficult to plan long-term evolution in research -> guided by breakthrough
% - Example of kid-size
This document is a proposal to replace the roadmap proposed in 2014 for the
RoboCup humanoid
league~\footnote{\url{http://www.robocuphumanoid.org/wp-content/uploads/HumanoidLeagueProposedRoadmap.pdf}}. Criticism
have emerged with respect to establishing a long-term schedule for evolution of
the rules in a research context where evolution is mainly guided by breakthrough
rather than by linear development. Some of the changes planned revealed to be
difficult to implement: the removal of the KidSize initially planned for 2020
seems now unrealistic since this league still received more applications than
TeenSize and AdultSize together in 2019. Finally, the relationship between
missing skills and research topics was not explicited.

% Aim of this document + events to gather opinions
In order to avoid facing this issues, a different approach is used. Rather than
establishing a long-term schedule, the document describes the targeted challenge
for the league, the steps required to tackle it. Upcoming changes in the next 5
years are described in more detail to allow teams to plan their developments. In
order to produce a roadmap satisfying for both the trustees and the league, a
workshop was run at IROS2018 and polls have been proposed to both of them. While
a consensus was obtained on most of the questions, this process has also shown
that for some propositions it is difficult to reach an agreement.

\subsection{Goals of the roadmap}

\subsection{Which path to follow?}
% First focus toward improving gameplay

\subsection{Structure of the document}

The structure of the documents is as
follows. Section~\ref{sec:ScientificProblematics} introduces the scientific
challenges related to the league and their importance in the games. The major
upcoming changes to the rules are presented in Section~\ref{sec:ShortTerm}. The
global evolution of the leagues is discussed in Section~\ref{sec:LongTerm}
through an event-triggered Roadmap and the history of changes which occured
since the creation of the league.

\section{\label{sec:ScientificProblematics}Scientific problematics}

\subsection{Motion control for humanoid robots}

% Very brief history reminder
The first humanoid robot to be built is the WABOT-1 which was completed in
1972. While WABOT-1 was already able to walk, controlling humanoid robots remain
a challenging problem today. From designing dynamic walking gait to running and
real-time adaptation of multi-objective motions, motion control of humanoid
robots is an active research topic.

% Focus of RoboCup : introduction
While bipedal locomotion is one of the main interests of the RoboCup humanoid
league, multiple dynamic motions are required such as standing-up and kicking
the ball. Given the adversarial context of the RoboCup, one of the key elements
is also the transition between different motions in order to reduce the time
required to achieve complex tasks. Finally, one the major concern from a
mechanical point of view are achieving satisfying performances while using
low-cost hardware and ensuring the robustness of the hardware despite the fact
that the robots are falling.

\subsubsection{Bipedal locomotion}

% Historically:
% ZMP -> Developed in 1970, still quite used today ``
% Walking, walking on uneven terrain, running

% RoboCup specificities:
% - controlable walk (

\subsubsection{Robustness and reduction of hardware cost}
% - Robots walking all game long (usually, only for demo)
% - Low-cost architecture (risk of damaging the robots, adult-size excepted)

\subsubsection{Motion learning}
% Several specific motions required (standing-up)

\subsubsection{Transition between motions}

\subsection{Embedded perception}

In order to act autonomously, robots require to be able to detect various
objects such as the ball, opponents and goals. Agregation of information and
filtering are also important in order to analyze the state of the game with a
high accuracy.

% Detection: from color based filters
% TODO: add references toward 'DNN articles' and tagger
% TODO: mention the semantic perception
Until RoboCup 2015 where white goals and white balls were introduced, most of
the object detection algorithms were based on color segmentation and shape
recognition. Since color were removed, most of the teams moved toward machine
learning methods such as deep neural network. In order to reduce the amount of
work required to label the data, tools have been developed to allow multiple
teams to mutualize their efforts toward producing dataset for the community. A
specificity of the league with respect to the field of computer vision is the
low computational power available, this constraint is shared by other robotics
applications such as drones.

% Localisation:
% - Currently mainly based on filters
% - Evolution 
Currently, most of the localization modules are based on particle filters which
allows to track multi-modal distributions. While these approaches have proved
their efficiency, recent advances in the SLAM domain suggest that localization
based on factor graphs rather than filtering are more
accurate~\cite{Strasdat2012}.

% Calibration topic
Most of the information used for localization relies on information provided by
the camera. Therefore, estimating accurately the orientation of the camera is
crucial to enhance the accuracy. During the last years, multiple teams have
started to work on automated calibration of geometric errors in the robot model
in order to increase the accuracy of the observation. This approaches are
generally based on the use of visual markers easily detectable such as Aruco
Tags~\cite{Garrido-Jurado2014}.


\subsection{Decision-making}

\subsection{Interaction with humans}

\subsection{Cost efficient hardware}

\section{\label{sec:ShortTerm}Short-term: upcoming changes}

\subsection{RoboCup 2019 changes}

The major rule changes for RoboCup 2019 are as following:

\begin{itemize}
\item Natural lightning
\item Throw-in, corner kicks, goal kicks % TODO describe more
\item Pick-up/penalties simplification
\item Requiring implementation of an emergency stop button, no constraints on design
\item Adult Size only
  \begin{itemize}
  \item The field size increases to $14 \times 9$ meters (previously $9 \times 6$ meters)
  \item Games are now played 2 vs 2
  \end{itemize}
\end{itemize}

There are also some minor adjustements:

\begin{itemize}
\item Time for repositioning during stoppage of game increased to 45 seconds (previously 30 seconds).
\item New phrasing regarding constraints on foot design.
\item The surface of teams markers is increased and it is allowed to place them on chest.
\end{itemize}

\subsection{RoboCup 2020 changes}

TODO: 
\begin{itemize}
\item Introduce Open Humanoid Leagues
\item Introduce Demo Events
\item New formula with Drop-In as seeding for Division A/Division B
\item Merge of Kid and Teen
\item Updates of technical challenges?
\end{itemize}

% TODO add emergency button

\section{\label{sec:LongTerm}Long-term: event triggered roadmap}

% TODO: Reminder of the reason pushing toward event-triggered roadmap

% TODO: Description of structure of the document

\subsection{Guiding principles}

% Events or measures are used to trigger the changes
Changes in the rules will be triggered by either research breakthrough or
continuous improvement of some skills. Those improvements will be ensured
through either measures of metrics during the game such as walking speed or
success at technical challenges. This will ensure that the rules improvements
are coherent with the capabilities of the robots in the league.

% Triggered changes apply with a delay, useful for teams and LOC
During the roadmap workshops and the polls, several teams requested to have
updates rules known sooner, in order to plan their development for the
competition. Similar issues arise for local organization with changes regarding
the size of the fields. In order to provide more time to teams and local
organizers while still allowing adaptation of the rules, we propose to separate
the evolution of the league in three categories depending on the number of years
required between announcement of the rule change and its application. Note that
we consider a 4 month period to allow the TC to implement the changes inside the
rulebook.
\begin{description}
\item[1Y:] Changes which can be done from one competition to the other without
  requiring previous announcement. This is limited to rule changes which have
  low impact on the robot design and do not require additional space for the
  competition. These changes have to be announced at least 8 months prior
  to the competition.
\item[2Y:] Changes which might require significant hardware investment for the
  teams or important modifications for the local organizers. These changes have
  to be announced at least 20 months prior implementation.
\item[5Y:] Major changes which require drastic modifications on robot design.
  These changes shave to be announced at least 56 months prior to the
  competition.
\end{description}

% TODO: Is change based on 'best team', 'n best teams' or something else?

\subsection{Metrics used}
% TODO: fill it

\subsection{Detailed rule changes}

\subsubsection{Field evolution}
The dimensions of the goal is a crucial aspect to ensure that scoring and
defending goals are both possible. It makes sense that the ratio between the
field width and the goal width is similar to the ratio in human soccer, but
depending on kick accuracy and capacity of diving for goalies, it might be
required to have smaller or larger goals.

%TODO: add recap diagram with transition elements

\subsubsection{Handlers}

\subsection{League status}
% TODO: Separate diagram in 2/3 different diagrams? Skills and rules?
% TODO: Add notes on the level
\begin{sidewaysfigure}
% This document contains a tentative of dependency diagram for the Humanoid League roadmap

\tikzstyle{element}=[rectangle, draw=black, rounded corners, text centered,
  anchor=north, text width=2cm, minimum height= 1cm]
\tikzstyle{transitionNote}=[rectangle, right, rounded corners, %text left,
  text width=3cm, minimum height= 1cm]
\tikzstyle{dependency} = [->, thick]

\begin{tikzpicture}[node distance=1cm, every node/.style={scale=0.6}]
  % Locomotion
  \node (Walking) [element] {Walking};
  \node (Running) [element, below=of Walking] {Running};
  \draw [dependency] (Walking) -- (Running);
  % Falling column
  % TODO: Still missing: Diving (goalie)
  \node (StandingUp) [element,right=of Walking] {Falling and standing up};
  % TODO: missing ball manipulation (hands + foot)
  % Kicking (add the transition part (kick from walking / kick from running))
  % Add the high kick?
  \node (StaticKick) [element,right=of StandingUp] {Static Kick};
  \node (RollingKick) [element, below=of StaticKick] {Kick from rolling ball};
  \node (FlyingKick) [element, below=of RollingKick] {Kick from flying ball};
  \draw [dependency] (StaticKick) -- (RollingKick);
  \draw [dependency] (RollingKick) -- (FlyingKick);
  % Perception
  \node (BallDetection) [element,right=of StaticKick] {Ball detection};
  \node (RollingBall) [element,below=of BallDetection] {Rolling ball tracking};
  \node (FlyingBall) [element,below=of RollingBall] {Flying ball tracking};
  \draw [dependency] (BallDetection) -- (RollingBall);
  \draw [dependency] (RollingBall) -- (FlyingBall);
  % Environment information
  \node (SidePoles) [element,right=of BallDetection] {Side Poles};
  \node (ColorCoded) [element,below=of SidePoles] {Colored objects};
  \node (HumanLike) [element,below=of ColorCoded] {No colors};
  \draw [dependency] (SidePoles) edge node [transitionNote] {2012: Side poles\\2013: Yellow goals} (ColorCoded) ;
  \draw [dependency] (ColorCoded) edge node [transitionNote] {2015:\\White goals\\White ball} (HumanLike);
  % Ground type (Carpet, Turf, outdoor ground)
  \node (Carpet) [element,right=of SidePoles] {Carpet};
  \node (Turf) [element,below=of Carpet] {Artificial turf};
  \node (RealGrass) [element,below=of Turf] {Real grass};
  \draw [dependency] (Carpet) edge node [transitionNote] {2015: 30mm turf} (Turf);
  \draw [dependency] (Turf) -- (RealGrass);
  
  % Nb players
  \node (1player) [element,right=of Carpet] {1 vs 1};
  \node (2players) [element,below=of 1player] {2 vs 2};
  \node (3players) [element,below=of 2players] {3 vs 3};
  \node (4players) [element,below=of 3players] {4 vs 4};
  \node (6players) [element,below=of 4players] {6 vs 6};
  \node (8players) [element,below=of 6players] {8 vs 8};
  \node (11players) [element,below=of 8players] {11 vs 11};
  \draw [dependency] (1player)  -- (2players);
  \draw [dependency] (2players) -- (3players);
  \draw [dependency] (3players) -- (4players);
  \draw [dependency] (4players) -- (6players);
  \draw [dependency] (6players) -- (8players);
  \draw [dependency] (8players) -- (11players);
  

  % Cross category dependencies
  \draw [dependency] (RollingBall) -- (RollingKick);
  \draw [dependency] (FlyingBall) -- (FlyingKick);
\end{tikzpicture}

%TODO: add removal of magnetometer

\end{sidewaysfigure}


\section{Evolution of the document}

\begin{itemize}
\item Status of the 'long-term' part updated yearly to assess progress
\item Status of the 'short-term' part updated every year or every two years
\item Scientific problematics: updated every 5 years
\end{itemize}

% TODO: Summarizing rule changes?
%\section{Past evolutions}

\newpage

\printbibliography


\end{document}
